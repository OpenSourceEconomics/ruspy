\documentclass[a4paper,12pt,bold]{scrartcl}

\renewcommand{\baselinestretch}{1.3}\normalsize
\newcommand{\vect}[1]{\mathbf{#1}}
\newcommand{\thin}{\thinspace}
\newcommand{\thick}{\thickspace}
\newcommand{\N}{\mathcal{N}}	%Normal Distribution
\newcommand{\U}{\mathrm{U}}	%Uniform Distribution
\newcommand{\D}{\mathrm{D}}	%Dirichlet Distribution
\newcommand{\W}{\mathrm{W}}	%Wishart Distribution
\newcommand{\E}{\mathrm{E}}		%Expectation
\newcommand{\Iden}{\mathbb{I}}	%Identity Matrix
\newcommand{\Ind}{\mathrm{I}}	%Indicator Function

\newcommand{\bs}{\boldsymbol}
\newcommand{\var}{\mathrm{var}\thin}
\newcommand{\plim}{\mathrm{plim}\thin}
\newcommand{\cov}{\mathrm{cov}\thin}
\newcommand\indep{\protect\mathpalette{\protect\independenT}{\perp}}
\def\independenT#1#2{\mathrel{\rlap{$#1#2$}\mkern5mu{#1#2}}}
\usepackage{bbm}
%\usepackage{endfloat}
\renewcommand{\vec}[1]{\mathbf{#1}}



\parindent0pt
\usepackage{algpseudocode,tabularx,ragged2e}
\newcolumntype{C}{>{\centering\arraybackslash}X} % centered "X" column
\newcolumntype{L}{>{\arraybackslash}X} % centered "X" column

\usepackage{algorithmicx}
\usepackage{graphicx}
\usepackage{afterpage}

\usepackage{algorithm}
\graphicspath{{../code/}}

\usepackage{float}
\usepackage[section]{placeins}
\usepackage{apacite}
\usepackage{booktabs}
\usepackage{epigraph}
\usepackage[sans]{dsfont}
\usepackage[round,longnamesfirst]{natbib}
\usepackage{bm}																									%matrix symbol
\usepackage{setspace}																						%Fu�noten (allgm.
\usepackage[colorlinks = true,
            linkcolor = blue,
            urlcolor  = blue,
            citecolor = blue,
            anchorcolor = blue]{hyperref}%Zeilenabst�nde)
\usepackage{threeparttable}
\usepackage{lscape}																							%Querformat
\usepackage[latin1]{inputenc}
													%Umlaute
\usepackage{graphicx}
\usepackage{amsmath}
\usepackage{amssymb}
\usepackage{fancybox}																						%Boxen und Rahmen
\usepackage{appendix}
\usepackage{listings}
\usepackage{xr}
\usepackage{pdflscape}

\usepackage{enumerate}
\usepackage[labelfont=bf]{caption}
																		%EURO Symbol
\usepackage{tabularx}
\usepackage{longtable}
\usepackage{subfig,float}																				%Mehrseitige Tabellen
\usepackage{color,colortbl}																			%Farbige Tabellen
\usepackage[left=3cm, right=2cm, top=2cm, bottom=2.5cm]{geometry} %Seitenr�nder
%\usepackage[normal]{caption2}[2002/08/03]												%Titel ohne float - Umgebung
\definecolor{lightgrey}{gray}{0.95}	%Farben mischen
\definecolor{grey}{gray}{0.85}
\definecolor{darkgrey}{gray}{0.80}

\newcommand{\mc}{\multicolumn}
\usepackage{rotating}

\usepackage{tikz}
\usetikzlibrary{positioning}
\usepackage[export]{adjustbox}

\usepackage{caption}
\captionsetup[figure]{labelfont=bf}

\usepackage{url}  % Used for linebreaks in verbatim statements
\usepackage{multirow}

\newtheorem{Definition}{Definition}
\newtheorem{Remark}{Remark}
\newtheorem{Lemma}{Lemma}
\newtheorem{Theorem}{Theorem}
\newtheorem{Assumption}{Assumption}
\newtheorem{Excercise}{Excercise}
\newtheorem{Result}{Result}
\newtheorem{Proposition}{Proposition}
\newtheorem{Prediction}{Prediction}
\newtheorem{Solution}{Solution}
\newtheorem{Problem}{Problem}

\setlength{\skip\footins}{1.0cm}
\deffootnote[1em]{1.1em}{0em}{\textsuperscript{\thefootnotemark}}
\renewcommand{\arraystretch}{1.05}

\DeclareMathOperator*{\argmin}{arg\,min}
\DeclareMathOperator*{\argmax}{arg\,max}




\newenvironment{boenumerate}
{\begin{enumerate}\renewcommand\labelenumi{\textbf{(\theenumi)}}}
{\end{enumerate}}
\makeatletter
\newenvironment{manquotation}[2][2em]
  {\setlength{\@tempdima}{#1}%
   \def\chapquote@author{#2}%
   \parshape 1 \@tempdima \dimexpr\textwidth-2\@tempdima\relax%
   \itshape}
  {\par\normalfont\hfill--\ \chapquote@author\hspace*{\@tempdima}\par\bigskip}



\setkomafont{author}{\scshape}
\usepackage{blindtext}

\title{Hackathon 19.12.: Optimal replacement of GMC bus engines: An empirical
model of Harold Zurcher}
\author{Maximilian Blesch}
\date{\today}



\begin{document}
\maketitle

\section{Introduction}
This handout will give a short description of the initial paper my thesis will work on. In my thesis I will compare the classic optimal decision making (Bellman equation) with robust decision making as described in Chapter 13 of \cite{Ben-Tal.2009}. A perfect setting to illustrate the power and the numerical implementation of this method is \cite{Rust.1987}, a paper on the optimal decision rule of bus engine replacements. After a quick introduction of the data, I will continue with the model that according to Rust describes the decision rule.

\section{Sample and data selection}
The agent in Rust's paper is the maintenance manager of the Madison (Wisconsin,US) Metropolitan Bus Company, Harold Zurcher. He provides Rust with a sample of 162 buses from December 1974 to May 1985. The data consists of monthly observations on the odometer readings for each bus, plus data on the date and odometer readings at which a bus engine was replaced. Furthermore Rust divides these 162 buses by model and date of purchase into 8 groups. Across these groups he conducts several analyses to find a homogeneous sample in regards to usage and timing of replacement. Group 4, the 1975 GMC 5308A, is one of the main workhorses of the company and has a fairly homogeneous usage during the 10 years of observation and with 37 buses also offers a decent amount of data. While he conducts a variety of analyses with various data sets, he uses the group 4 data set for all analyses. Therefore I will implement all of my analyses initially with this group.

\section{Optimal Decision Model}
Rust chooses the following setting: First of all he assumes that there is complete independence of the decisions made for each bus and thus refrains from using superscripts indicating the buses. In every month (from now on called period) the agent, Harold Zurcher, has the choice to either replace (\(i_t = 1 \)) or to maintain (\(i_t = 0\)) a bus. The agent chooses his action with the aim to maximize the current value of the bus (Bellman equation):
  \begin{equation}
    V_{\theta}(x_t, \epsilon_t) = \max_{i_t \in \{0, 1\}} [ u(x_t, i_t, \theta) + \epsilon_t (i_t) + \beta EV_{ \theta }( x_t, \epsilon_t, i_t) ]
  \end{equation}
with
\begin{equation}
  u(x_t,i_t, \theta)=
  \begin{cases}
  -c(x_t, \theta_1)\qquad \qquad  \textbf{if} \quad i_t=0 \\
  -[RC + c(0, \theta_1)]\quad \textbf{if} \quad i_t=1
  \end{cases}
\end{equation}
and
\begin{equation}
  EV_{\theta}(x_t, \epsilon_t, i_t) = \int_{\gamma}\int_{\eta} V_{\theta}(\gamma, \eta)p(d\gamma, d\eta | x_t, \epsilon_t, i_t, \theta_2, \theta_3),
\end{equation}

  \bigskip
where \\
\begin{table}[htbp]
    \centering % to have the caption near the table
    \begin{tabular}{l c p{10cm} }
        $x_t$ & : & Mileage on the odometer of the bus in period $t$\\
        $\varepsilon_t$ & : & Unobserved information for each decision in period $t$\\
        $\mathbf{EV}_{\theta}$ & : & Future expected value of each decision\\
    \end{tabular}
\end{table}

and $\theta = \left(\theta_1, \theta_2, \theta_3, \mathbf{RC}, \beta \right)$ is the vector of the unknown variables to be estimated.



\begin{table}[htbp]
    \centering % to have the caption near the table
    \begin{tabular}{l c p{10cm} }
        $\theta_1$ & : & Cost parameter\\
        $\theta_2$ \& $\theta_3$ & : & Factors that determines the transition probabilities\\
        $\mathbf{RC}$ & : & Replacement costs of a bus engine\\
        $\beta$ & : & Discount factor\\
    \end{tabular}
\end{table}

Further, he assumes conditional independence of the transition probabilities of the error term and the development of mileage (A6 of \cite{Rust.1988}):

\begin{Assumption}
Conditional Independence Assumption(CI):\\
The transition density of the controlled process \(\{x_t, \epsilon_t\}\) factors as
\begin{equation}
  p(x_{t+1}, \epsilon_{t+1} | x_t, \epsilon_t, i_t, \theta_2) = q(\epsilon_{t+1} | x_{t+1}, \theta_2) p(x_{t+1} | x_t, i_t, \theta_3)
\end{equation}
\end{Assumption}
This theorem involves two restrictions. First, \(x_{t+1}\) is a sufficient statistic for \(\epsilon_{t+1}\), which implies that any statistical dependence between \(\epsilon_t\) and \(\epsilon_{t+1}\) is transmitted entirely through the vector \(x_{t+1}\). Second, the probability density of \(x_{t+1}\) depends only on \(x_t\) and not on \(\epsilon_t\). The payoff of (CI) is twofold. First, (CI) implies that \(EV_{\theta}\) is not a function of \(\epsilon_t\), so that the required choice probabilities will not require integration over the unknown function \(EV_{\theta}\). Second, (CI) implies that \(EV_{\theta}\) is a fixed point of a separate contraction mapping. For the density of \(\epsilon_t\), Rust assumes means \((0,0)\) and variances \( ( \pi / 6, \pi / 6) \).\medskip \\
Due to computational reasons, Rust discretized the mileage of the buses into 90 states, each representing 5000 miles. Therefore $x_t$ represents the state of the bus in each period. Furthermore, as the monthly mileage increase in the data is never larger than 15,000, the transition probabilities reduce to a multinomial distribution on \{1, 2, 3\}.\\
Together with (CI) and the notation \(EV_{ \theta }( x_t, 1) = EV_{ \theta }( 0, 0) =: EV(0)\) and \(EV_{ \theta }( x_t, 0) =: EV(x_t)\) for all \(x_t\), Rust derives the following equation for the expected future value:
\begin{equation}
  EV_{\theta}(x_t) = \sum_{j \in \{1, 2, 3\}} p_j * \ln\{ \sum_{i_t \in \{0, 1\}} \exp[u(x_t, i_t, \theta_1, RC) + \beta EV_{\theta}(i_t * (x_t + j))]\}
\end{equation}
with \(p_j\) the transition probabilities to \(j \in \{1,2,3\}\) and for the choice probabilities:
\begin{equation}
P(i_t | x_t, \theta) = \frac{\exp[u(x_t, i_t, \theta_1, RC) + \beta EV_{\theta} (i_t * x_t)]}{ \sum_{j \in \{0, 1\}}\exp[u(x_t, j, \theta_1, RC) + \beta EV_{\theta} (j * x_t)]}
\end{equation}
The likelihood function for the estimation of \(\theta\) can be split up into two separate functions:
\begin{equation}
  l^1(x_1, .....,  x_T, i_1, ...., i_T | x_0, i_0, \theta) = \prod_{t = 1}^T p(x_t | x_{t-1}, i_{t-1}, \theta_3)
\end{equation}
for the transition probabilities and
\begin{equation}
  l^2(x_1, .....,  x_T, i_1, ...., i_T | \theta) = \prod_{t = 1}^T P(i_t | x_t, \theta_1, RC, \theta_3)
\end{equation}
for the cost parameters $RC$ and \(\theta_1\).
For a detailed proof of the above please, refer to \cite{Rust.1987} and \cite{Rust.1988}.

\newpage
\bibliographystyle{apacite}
\bibliography{bibliography/literature.bib}

\end{document}
